\documentclass[a4paper,11pt]{article}

\usepackage[T1]{polski}
\usepackage[utf8]{inputenc}
\usepackage{graphicx}
\usepackage{amsmath}
\usepackage{makecell}
\usepackage{listings}


\hoffset=-3.0cm                 
\textwidth=18cm                
\evensidemargin=0pt
\voffset=-3cm                  
\textheight=27cm                
\setlength{\parindent}{0pt}             
\setlength{\parskip}{\medskipamount}    
\raggedbottom                           

\title{Wykorzystanie MS SQL i pakietów MS Business Intelligence \\do budowy aplikacji}
\author{Smoliński Mateusz}
\date{kwiecień - czerwiec 2022}

\begin{document}
\maketitle
\tableofcontents
\newpage
\section{Zakres ćwiczenia}
\begin{center}
    \begin{tabular}{|c|c|}
        \hline
        Numer & Opis wymagania \\ \hline
        W001 & \makecell {Import plików z danymi sprzedaży, dostaw i ręcznie obliczonego stanu sklepu \\
        (napisanego w Excelu, zapisanego do Unicode txt)}\\ \hline
        W001a & \makecell{Dane sprzedaży zawierają: numer rachunku, datę, ilość/liczbę produktów, \\ 
        kod produktu, cenę brutto, cenę netto, stawkę VAT}\\ \hline
        W001b & \makecell{Dane dostaw zawierają: kod dostawcy, datę, ilość/liczbę produktów, \\
        kod i nazwę produktu, kod i nazwę producenta, cenę hurtową}\\ \hline
        W002 & \makecell{Powstaje słownik produktów i producentów}\\ \hline
        W003 & \makecell{Obliczanie różnicy między teoretycznym, a rzeczywistym stanem kasy}\\ \hline
        W004 & \makecell{Wstawienie nowego pliku z ręczenie obliczonym stanem poprzedzone jest usunięciem \\
        danych ze starego.}\\ \hline
        W005 & \makecell{Generowanie raportu dotyczących pojedynczego produktu}\\ \hline
        W006 & \makecell{Wszystkie kwoty w PLN (jedna waluta)}\\ \hline
        W007 & \makecell{Produkty nie mają stałych cen, na różnych dokumentach mogą występować inne}\\ \hline

        


    \end{tabular}
\end{center}
Zrezygnowano z punktów dotyczących porównania stanu w bazie ze stanem ręcznie policzonym i 
porównywania pieniędzy.
\section{Opis problemu}
Aplikacja zajmuje się przechowywaniem i przeliczaniem danych dotyczących stanu magazynu sklepu, 
na podstawie wgrywanych plików z dostaw lub danych sprzedaży i informacji wpisanych ręcznie.
Można sprawdzić stan liczbowy poszczególnych produktów i zmiany w ich liczebności. 
Została wykonana w środowisku Java przy użyciu Spring Boota i Vaadin.

\section{Analiza funkcjonalna}
\subsection{Wygrywanie pliku}
Aplikacja na głównym widoku posiada możliwość wgrania danych z pliku zapisanego do formatu Unicode txt.
Kolumny w pliku muszą być odpowiednio nazwane, żeby odczyt przebiegł pomyślnie.
Podczas wczytywania pliku automatycznie tworzeni są pojawiający się po raz pierwszy producenci, produkty,
dostawy i rachunki oraz aktualizowany jest ich stan. Plik nie jest zapisywany w bazie danych po jego odczytaniu.

\subsection{Dodanie elementów}
W ramach aplikacji, na każdym z dostępnych okien widoku (z wyjątkiem Stock - zapas) można dodać do bazy 
danych nowy rekord z podanymi danymi. Aplikacja potrzebuje otrzymać kody identyfikacyjne istniejących już 
elementów w bazie, którym podlega nowo dodany obiekt (np. produkt wymaga kodu producenta) oraz unikalny 
kod dodawanego obiektu, jeżeli taki jest potrzebny.
\subsection{Przeglądanie zawartości tabel}
Aplikacja posiada widoki do różnych tabel z bazy danych, gdzie wyświetlona jest przechowywana 
zawartość.

\subsection{Podgląd produktu}
W widoku Produkty można przejść do widoku informacji o produktach, który wyświetli zakupy i sprzedaże 
tego produktu przez sklep. 

\section{Struktura bazy}
Baza danych jest automatycznie tworzona przy pomocy Spring Boota. 
Zostały dodane obiekty odpowiadające relacjom many to many.
Tworzone są tabele o odpowiadających 
obiektom nazwach:
\begin{itemize}
    \item Producers - producenci
    \item Products - produkty, wymagają producenta
    \item Stock - zapas produktów, tworzone razem z produktem
    \item Delivery - dostawy do sklepu
    \item Purchases - many to many pomiędzy produktami i dostawami, wymaga obu
    \item Receipt - rachunki (zakupy kleintów)
    \item Sales - many to many pomiędzy produktami i rachunkami, wymaga obu
\end{itemize}
\section{Opis narzędzi i sposobu realizacji}
Spring Boot upraszcza pracę z bazami danych tworząc automatyczne przypisania obiektów Javy 
do tabel bazy danych. Ułatwia komunikację aplikacji z bazą danych. 
Vaadin pozwala tworzyć przejrzyste aplikacje w Javie przy użyciu przygotowanych 
komponentów bez potrzeby zajmowania się frontenem.
\section{Dokumentacja działania}
\subsection{Treść plików z danymi}
\begin{lstlisting}
numer	dostawa	data	kod produktu	nazwa produktu	kod producenta	nazwa producenta	ile	cena hurtowa
1	AX11	21.04.2022	3170512345	pe cola	PC341	Pe Company z. o.o.	480	3,78
2	AX11	21.04.2022	3170554823	pe zelki	PC341	Pe Company z. o.o.	100	3,68
3	AX11	21.04.2022	3170564424	pe sok jablkowy	PC341	Pe Company z. o.o.	120	2,83
4	BEL314	21.04.2022	2041412064	kajzerka mala z	PIEK02	Piekarnia Zima	60	0,33
5	BEL314	21.04.2022	2041454345	bagietka z	PIEK02	Piekarnia Zima	10	1,91
6	BEL314	21.04.2022	3182146375	chrupki serowe	SERG	Serowe Gub	60	2,41
7	BEL314	21.04.2022	3182146371	prazynki	SERG	Serowe Gub	60	2,35
\end{lstlisting}

\begin{lstlisting}
numer	rachunek	data	kod produktu	nazwa produktu	ile	netto	brutto	vat percent
1	220422101143	22.04.2022 10:11:43	3170512345	pe cola	2	3,65	4,49	23
2	220422101143	23.04.2022 10:11:43	3170564424	pe sok jablkowy	1	3,69	3,99	8
3	220422101143	24.04.2022 10:11:43	2041412064	kajzerka mala z	10	0,37	0,39	5
4	220422101621	24.04.2022 10:16:21	3182146371	prazynki	2	4,22	5,19	23
5	220422101621	24.04.2022 10:16:21	3170512345	pe cola	3	3,65	4,49	23
6	220422102302	24.04.2022 10:23:02	2041412064	kajzerka mala z	6	0,37	0,39	5
7	220422102302	24.04.2022 10:23:02	2041454345	bagietka z	1	2,56	2,69	5
8	220422102302	24.04.2022 10:23:02	3182146375	chrupki serowe	2	3,41	4,19	23
9	220422102302	24.04.2022 10:23:02	3182146371	prazynki	1	4,22	5,19	23
10	220422102651	24.04.2022 10:26:51	2041412064	kajzerka mala z	5	0,37	0,39	5
11	220422102930	24.04.2022 10:29:30	3170512345	pe cola	1	3,65	4,49	23
12	220422102930	24.04.2022 10:29:30	3170554823	pe zelki	2	3,65	4,49	23
13	220422103311	24.04.2022 10:33:11	3170512345	pe cola	1	3,65	4,49	23
14	220422103738	24.04.2022 10:37:38	3170564424	pe sok jablkowy	1	3,69	3,99	8
15	220422103738	24.04.2022 10:37:38	2041454345	bagietka z	2	2,56	2,69	5
\end{lstlisting}
\subsection{Przykładowe widoki}
\begin{center}
    \includegraphics[width=17cm]{fig1.png}
\end{center}
\begin{center}
    \includegraphics[width=17cm]{fig2.png}
\end{center}
\begin{center}
    \includegraphics[width=17cm]{fig3.png}
\end{center}





\section{Dalsze możliwości rozwoju}
\begin{itemize}
    \item Odczyt plików w innych formatach.
    \item Zapisywanie plików z danymi do bazy danych.
    \item Filtrowanie po atrybutach.
    \item Przekierowywanie do widoku obiektu przy kliknięciu na jego nazwę.
    \item Generowanie plików CSV.
    \item Porównanie zmian cen produktów.
\end{itemize}



\end{document}